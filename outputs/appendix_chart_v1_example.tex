% LaTeX Appendix: chart_v1 Test Case Example
% Required packages: \usepackage{listings} \usepackage{xcolor}
% For syntax highlighting: \lstset{language=Java, basicstyle=\ttfamily\footnotesize, keywordstyle=\color{blue}, commentstyle=\color{green!60!black}, stringstyle=\color{red}}

\newpage
\section{chart_v1 Test Case Example}
\label{app:chartv1_example}

This appendix shows an example test case from the chart_v1 test suite.

\begin{lstlisting}[language=Java, caption=Example JUnit test class from chart_v1, label=lst:chartv1_test]
/* =========================================================== * JFreeChart : a free chart library for the Java(tm) platform * =========================================================== * * (C) Copyright 2000-2009, by Object Refinery Limited and Contributors. * * Project Info: http://www.jfree.org/jfreechart/index.html * * This library is free software;
you can redistribute it and/or modify it * under the terms of the GNU Lesser General Public License as published by * the Free Software Foundation;
either version 2.1 of the License, or * (at your option) any later version. * * This library is distributed in the hope that it will be useful, but * WITHOUT ANY WARRANTY;
without even the implied warranty of MERCHANTABILITY * or FITNESS FOR A PARTICULAR PURPOSE. See the GNU Lesser General Public * License for more details. * * You should have received a copy of the GNU Lesser General Public * License along with this library;
if not, write to the Free Software * Foundation, Inc., 51 Franklin Street, Fifth Floor, Boston, MA 02110-1301, * USA. * * [Java is a trademark or registered trademark of Sun Microsystems, Inc. * in the United States and other countries.] * * -------------------------------- * CategoryLineAnnotationTests.java * -------------------------------- * (C) Copyright 2005-2009, by Object Refinery Limited and Contributors. * * Original Author: David Gilbert (for Object Refinery Limited);
* Contributor(s): -;
* * Changes * ------- * 29-Jul-2005 : Version 1 (DG);
* 23-Apr-2008 : Added testPublicCloneable() (DG);
* */
package org.jfree.chart.annotations.junit;
import java.awt.BasicStroke;
import java.awt.Color;
import java.io.ByteArrayInputStream;
import java.io.ByteArrayOutputStream;
import java.io.ObjectInput;
import java.io.ObjectInputStream;
import java.io.ObjectOutput;
import java.io.ObjectOutputStream;
import junit.framework.Test;
import junit.framework.TestCase;
import junit.framework.TestSuite;
import org.jfree.chart.annotations.CategoryLineAnnotation;
import org.jfree.chart.util.PublicCloneable;
/** * Tests for the \{
@link CategoryLineAnnotationTests
\}
class. */
public class CategoryLineAnnotationTests extends TestCase \{
/** * Returns the tests as a test suite. * * @return The test suite. */
public static Test suite() \{
return new TestSuite(CategoryLineAnnotationTests.class);
\}
/** * Constructs a new set of tests. * * @param name the name of the tests. */
public CategoryLineAnnotationTests(String name) \{
super(name);
\}
/** * Confirm that the equals method can distinguish all the required fields. */
public void testEquals() \{
BasicStroke s1 = new BasicStroke(1.0f);
BasicStroke s2 = new BasicStroke(2.0f);
CategoryLineAnnotation a1 = new CategoryLineAnnotation("Category 1", 1.0, "Category 2", 2.0, Color.red, s1);
CategoryLineAnnotation a2 = new CategoryLineAnnotation("Category 1", 1.0, "Category 2", 2.0, Color.red, s1);
assertTrue(a1.equals(a2));
assertTrue(a2.equals(a1));
// category 1 a1.setCategory1("Category A");
assertFalse(a1.equals(a2));
a2.setCategory1("Category A");
assertTrue(a1.equals(a2));
// value 1 a1.setValue1(0.15);
// ... (67 more lines truncated for brevity) ...
\end{lstlisting}

This test case demonstrates the structure of JUnit tests used in the chart_v1 dataset, including test methods, assertions, and setup code.
